\documentclass[slidestop]{beamer}

\author{Jeroen F.J. Laros}
\title{Binary file parsing}
\providecommand{\mySubTitle}{A general approach}
\providecommand{\myConference}{Development work discussion}
\providecommand{\myDate}{08-05-2017}
\providecommand{\myGroup}{}
\providecommand{\myDepartment}{Department of Human Genetics}
\providecommand{\myCenter}{Center for Human and Clinical Genetics}

\usetheme{lumc}

\begin{document}

% This disables the \pause command, handy in the editing phase.
%\renewcommand{\pause}{}

% Make the title slide.
\makeTitleSlide{\includegraphics[height=2cm]{bin}}

% First page of the presentation.
\section{Introduction}
\makeTableOfContents

\subsection{Problem description}
\begin{pframe}
  \begin{figure}[]
    \begin{center}
      \includegraphics[height=0.7\textheight]{cyrillic_1}
    \end{center}
    \caption{Cyrillic 2.1.3.}
  \end{figure}
\end{pframe}

\begin{pframe}
  Cyrillic, a pedigree editor.
  \begin{itemize}
    \item Cyrillic was no longer usable (supported up to Windows XP).
    \item Proprietary binary file format (not FAIR).
    \item We have hundreds of pedigrees we can no longer access.
  \end{itemize}
  \bigskip

  Goals:
  \begin{itemize}
    \item Portable, open source pedigree editor using open standards.
    \item Can be used online.
    \item Can be embedded.
    \begin{itemize}
      \item Miracle.
      \item LOVD.
    \end{itemize}
  \end{itemize}
\end{pframe}

\subsection{Nested}
\begin{pframe}
  \begin{figure}[]
    \begin{center}
      \includegraphics[width=\textwidth]{nested}
    \end{center}
    \caption{Nested online editor.}
  \end{figure}
\end{pframe}

\section{Reverse engineering}
\subsection{Difference of hexadecimal dumps}
\begin{pframe}
  \begin{figure}[]
    \begin{center}
      \includegraphics[height=0.7\textheight]{fam_re}
    \end{center}
    \caption{A Cyrillic FAM file.}
  \end{figure}
\end{pframe}

\subsection{Standard approach}
\begin{pframe}
  \begin{lstlisting}[language=Python, caption={Ad hoc parser snippet.}]
    output['weight'] = bin_to_int(handle.read(2))
    output['age'] = bin_to_int(handle.read(2))
  \end{lstlisting}

  Drawbacks:
  \begin{itemize}
    \item Knowledge of the file format is in code (no documentation).
    \item Single purpose (only reading, one programming language).
  \end{itemize}
\end{pframe}

\subsection{A general approach}
\begin{pframe}
  \begin{figure}[]
    \begin{center}
      \includegraphics[height=0.3\textheight]{bin_parser_hld}
    \end{center}
    \caption{High level design of a general binary parser.}
  \end{figure}

  Specification separate from implementation.
  \begin{itemize}
    \item Specification of the basic types.
    \item Specification of the global file structure.
  \end{itemize}
\end{pframe}

\section{General binary file parser}
\subsection{Types}
\begin{pframe}
  Every basic type is processed in the following way:
  \begin{itemize}
    \item Data acquirement.
    \begin{itemize}
      \item Fixed number of bytes, delimited or a combination.
    \end{itemize}
    \item Data interpretation.
    \begin{itemize}
      \item Passing the data to a specific function.
    \end{itemize}
  \end{itemize}
  \bigskip

  The type \lstinline{short} ($2$ bytes passed to the \lstinline{int}
  function).
  \begin{lstlisting}[language=none, caption={Simple types definition.}]
    short:
      size: 2
      function:
        name: int
  \end{lstlisting}
\end{pframe}

\begin{pframe}
  A more complicated type.
  \begin{lstlisting}[language=none, caption={Less simple types definition}].
    comment:
      delimiter:
        - 0x00
      function:
        name: text
        args:
          split:
            - 0x09
  \end{lstlisting}

  Zero delimited string that is passed to the function \lstinline{text}, this
  function also receives the argument "\lstinline{split=[0x09]}".
\end{pframe}

\subsection{Structure}
\begin{pframe}
  We can now use the defined types in the structure definition.
  \begin{lstlisting}[language=none, caption={Structure snippet.}]
    - name age
      type short
    - name free_text
      type comment
  \end{lstlisting}

  Output:
  \begin{lstlisting}[language=none, caption={Output snippet.}]
    age: 51
    free_text: 'Blue eyes, yellow teeth.'
  \end{lstlisting}
\end{pframe}

\begin{pframe}
  Structures can be nested.
  \begin{itemize}
    \item \lstinline{for}-loops.
    \item \lstinline{while}-loops.
    \item Conditional statements (\lstinline{if}-clause).
  \end{itemize}
  \bigskip

  Operators:
  \begin{itemize}
    \item Logical (\lstinline{not}, \lstinline{and}, \lstinline{or}, \ldots).
    \item Arithmetical (\lstinline{eq}, \lstinline{lt}, \lstinline{ge},
      \ldots).
  \end{itemize}
\end{pframe}

\begin{pframe}
  Simple looping over a sub-structure.
  \begin{lstlisting}[language=none, caption={Simple for-loop.}]
    - name: fixed_size_list
      for: 2
      structure:
        - name: item
        - name: value
          type: int
  \end{lstlisting}

  The \lstinline{structure} keyword can also be used separately.
  \begin{itemize}
    \item Group certain variables in a logical way.
  \end{itemize}
\end{pframe}

\begin{pframe}
  Here we loop until \lstinline{value} is not equal to $2$.
  \begin{lstlisting}[language=none, caption={More complicated do-while-loop.}]
    - name: variable_size_list
      do_while:
        operands:
          - value
          - 2
        operator: ne
      structure:
        - name: item
        - name: value
          type: int
  \end{lstlisting}
\end{pframe}

\section{Implementations}
\subsection{Programming libraries}
\begin{pframe}
  Python implementation:
  \begin{itemize}
    \item Library: $496$ lines.
    \begin{itemize}
      \item Base class: $142$ lines, reader: $202$ lines, writer: $135$ lines.
    \end{itemize}
    \item Type functions: $195$ lines.
  \end{itemize}
  \bigskip

  JavaScript implementation:
  \begin{itemize}
    \item Library: $691$ lines.
    \begin{itemize}
      \item Base class: $174$ lines, reader: $273$ lines, writer: $174$ lines.
    \end{itemize}
    \item Type functions: $323$ lines.
  \end{itemize}
  \bigskip
\end{pframe}

\begin{pframe}
  Both implementations have been packaged (PyPI, npm), install with one
  command.
  \bigskip

  Both implementations have a command line interface.
  \begin{lstlisting}[language=none, caption={Reader command line interface.}]
    bin_parser read my_file.bin \
      structure.yml types.yml my_file.yml
  \end{lstlisting}

  \begin{lstlisting}[language=none, caption={Writer command line interface.}]
    bin_parser write my_file.yml \
      structure.yml types.yml my_file.bin
  \end{lstlisting}
\end{pframe}


\subsection{Adding new types}
\begin{pframe}
  \vspace{-0.5cm}
  \begin{lstlisting}[language=none, caption={Adding a type in Python.}]
    from bin_parser import BinReadFunctions

    class Invert(BinReadFunctions):
        def inv(self, data):
                return data ^ 0xff
  \end{lstlisting}

  \vspace{-0.5cm}
  \begin{lstlisting}[language=none, caption={Adding a type in JavaScript.}]
    var Functions = require('functions');

    function Invert() {
      this.inv = function(data) {
        return data ^ 0xff;
      };

      Functions.BinReadFunctions.call(this);
    }
  \end{lstlisting}
\end{pframe}

\section{Examples}
\subsection{FAM parser}
\begin{pframe}
  Structure: $259$ lines, types: $183$ lines.
  \begin{lstlisting}[language=none, caption={Structure snippet.}]
    - name: members
      for: number_of_members
      structure:
        - name: surname
        - name: forenames
        - name: address
          type: comment
        - name: spouses
          for: number_of_spouses
          structure:
            - name: id
              type: int
            - name: flags
              type: relationship
            - name: name
  \end{lstlisting}
\end{pframe}

\begin{pframe}
  Resulting output in YAML format.
  \begin{lstlisting}[language=none, caption={Output snippet.}]
    members:
      - surname: 'Gambolputty'
        forenames: 'Johann'
        address: 'Ulm'
        spouses:
          - flags:
              consanguineous: false
              divorced: false
              informal: false
              separated: false
            id: 2
            name: ''
  \end{lstlisting}
\end{pframe}

\begin{pframe}
  \begin{figure}[]
    \begin{center}
      \includegraphics[height=0.7\textheight]{cyrillic_1}
    \end{center}
    \caption{Original pedigree.}
  \end{figure}
\end{pframe}

\begin{pframe}
  We can edit the YAML file and convert it back to FAM format.
  \begin{lstlisting}[language=none, caption={Updating date of death field.}]
    date_of_birth: unknown
    date_of_death: defined
    father_id: 0
    flags:
      blood: false
      cells: false
      committed_suicide: false
      dna: false
      hide_info: false
      loop_breakpoint: false
  \end{lstlisting}
\end{pframe}

\begin{pframe}
  \begin{figure}[]
    \begin{center}
      \includegraphics[height=0.7\textheight]{cyrillic_2}
    \end{center}
    \caption{Edited pedigree.}
  \end{figure}
\end{pframe}

\subsection{Prince of Persia}
\begin{pframe}
  \begin{figure}[]
    \begin{center}
      \includegraphics[height=0.7\textheight]{prince}
    \end{center}
    \caption{Level 1.}
  \end{figure}
\end{pframe}

\begin{pframe}
  \begin{figure}[]
    \begin{center}
      \includegraphics[height=0.7\textheight]{prince_hs}
    \end{center}
    \caption{Original high scores.}
  \end{figure}
\end{pframe}

\begin{pframe}
  Like before, we convert, edit and convert back.
  \begin{lstlisting}[language=none, caption={Output snippet.}]
    entries:
      - minutes: 31
        name: Jeroen
        seconds: 2
      - minutes: 29
        name: Hahahahah
        seconds: 55
      - minutes: 29
        name: Jeroen
        seconds: 53
    ...

    number_of_entries: 6
  \end{lstlisting}
\end{pframe}

\begin{pframe}
  \begin{figure}[]
    \begin{center}
      \includegraphics[height=0.7\textheight]{prince_hs2}
    \end{center}
    \caption{Edited high scores.}
  \end{figure}
\end{pframe}

% Make the acknowledgements slide.
\makeAcknowledgementsSlide{
  \begin{tabular}{ll}
    Martijn Vermaat\\
    Zuotian Tatum
  \end{tabular}
}

\end{document}
